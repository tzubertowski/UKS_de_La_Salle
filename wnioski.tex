\documentclass{scrartcl}
\usepackage[utf8]{inputenc}
\usepackage{polski}
\usepackage{amsmath}
\usepackage{graphicx}

\title{UKS de La Salle.}
\subtitle{Modernizacja serwisu.}
\author{Tomasz Szymanek
\and
Tomasz Żubertowski}

\date{2013-2014}

\usepackage{natbib}
\usepackage{graphicx}

\begin{document}

\maketitle

\section{UKS de La Salle}
W ramach projektu zespołowego na Uniwersytecie Gdańskim chcemy podjąć się modernizacji serwisu (we współpracy z JIT Solutions):
http://uksdelasalle.pl/
\section{Aspekty prawne}
\subsection{Fonty/Czcionki}
Aby nie przeciążać hostingu i zniwelować problem roszczeń co do praw autorskich d/t czcionek planujemy bazować jedynie na czcionkach dostępnych na licencji Open Source, najprawdopodobniej wyłącznie z serwisu: 
http://www.google.com/fonts
\subsection{Zdjęcia, obrazy, stocki}
Sprawa z obrazami ma się prościej aniżeli z czcionkami. Wszystkie zdjęcia i obrazy będą pochodzić ze źródeł uczniowskiego klubu sportowego UKS de La Salle. Ewentualne grafiki użyte w serwisie będą wygenerowane albo przy użyciu css, albo będą posiadały licencję Open Source.
\subsection{Technologia/narzędzia pracy}
Nie będziemy korzystać z gotowych rozwiązań w postaci skryptów, czy całych silników cms ( : ) ). Ponadto wszystkie narzędzia: edytory tekstu, edytory grafiki, których użycie zaplanowaliśmy są darmowe do użytku komercyjnego, bądź też posidamy do nich licencje.

\section{Wykonanie, koncept}

\begin{figure}[h!]
  \caption{Porównanie czasów działania.}
  \centering
    \includegraphics[width=0.7\textwidth]{czas.png}
\end{figure}

\bibliographystyle{plain}
\bibliography{references}
\end{document}
